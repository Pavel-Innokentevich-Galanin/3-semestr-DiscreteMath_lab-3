\begin{ESKDtitlePage}
    % Содержание
    \tableofcontents 
    \newpage
\end{ESKDtitlePage}

\begin{center}
    \textbf{Лабораторная работа №\titlePageLabNumber}
\end{center}

\textbf{Тема}: "\titlePageTopic".

\textbf{Цель}: "...".

\begin{center}
    \textbf{Ход работы}:
\end{center}

\newpage

% \subsection{Графы}

% \begin{enumerate}
%     \item Построить матрицу смежности и инцидентности для заданного графа. Изобразить граф.
%     \item По матрице смежности (инцидентности) для каждой из вершин вычислить ее степень. 
%     \item Используя поиск в глубину и поиск в ширину написать программу, определяющую число компонент связности графа. Методы представляются в виде отдельных функций (или классов).
%     \item Построить деревья поиска в ширину и глубину. 
%     \item Из заданного неориентированного графа построить произвольным образом ориентированный граф (добавить к каждому ребру стрелку). Для полученного таким образом ориентированного графа построить матрицу смежности и инцидентности. 
%     \item Из заданного неориентированного графа построить произвольным образом псевдограф.
%     \item Варианты заданий указаны в таблице 1.
%     \item В таблице граф задан списком ребер, например, запись (1,2)означает, что существует ребро, соединяющее вершину 1 с вершиной 2. 
% \end{enumerate}

% \begin{table}[!htp]
%     \caption{Пример}

%     \label{tabular:670}

%     \begin{center}
%         \begin{tabular}{|p{1cm}||p{1cm}|p{1cm}|p{13cm}|}
% №  & Кол. вершин & Кол. ребер & Задание графа  \\ \hline \hline
% 1. & 9           & 10         & $(1,2), (1,4), (1,6), (2,3), (2,5), (3,5), (4,5), (4,6), (7,8), (7,9)$ \\ \hline
% 2. & 6           & 8          & $(1,2), (1,4), (1,5), (1,6), (2,3), (3,4), (4,5), (5,6)$ \\ \hline
% 3. & 8           & 8          & $(1,2), (1,3), (2,3), (3,4), (3,5), (4,5), (5,6), (7,8)$ \\ \hline
% 4. & 8           & 6          & $(1,6), (1,8), (2,5), (3,6), (3,8), (4,5)$ \\ \hline
% 5. & 6           & 10         & $(1,2), (1,3), (1,4), (2,3), (2,4), (3,4), (3,5), (4,5), (4,6), (5,6)$ \\ \hline
% 6. & 9           & 8          & $(1,2), (1,3), (1,4), (2,3), (2,4), (3,4), (5,6), (6,7), (8,9)$ \\ \hline
% 7. & 8           & 7          & $(1,4), (2,4), (2,6), (3,6), (4,5), (5,6), (7,8)$ \\ \hline
% 8. & 6           & 10         & $(1,2), (1,3), (2,3), (2,4), (2,5), (3,4),(3,5), (4,5), (4,6), (5,6)$ \\ \hline
% 9. & 8           & 8          & $(1,2), (1,3), (1,4), (2,3), (2,4), (3,4), (5,7), (6,8)$ \\ \hline
% 10.& 6           & 8          & $(1,2), (1,4), (2,3), (2,4), (3,4), (3,6), (4,5), (5,6)$ \\ \hline
% 11.& 8           & 6          & $(1,3), (2,3), (3,4), (4,5), (5,6), (7,8)$ \\ \hline
% 12.& 7           & 12         & $(1,2), (1,3), (1,4), (1,6), (2,3), (2,4), (2,6), (2,7), (3,4), (4,5), (5,6), (6,7)$ \\ \hline
% 13.& 9           & 10         & $(1,2), (1,3), (2,3), (3,4), (4,5), (4,6), (3,4), (7,8), (7,9), (8,9)$ \\ \hline
% 14.& 11          & 10         & $(1,2), (2,4), (2,5), (3,4), (4,7), (5,6), (5,7), (7,8), (9,10), (9,11)$ \\ \hline
%         \end{tabular}
%     \end{center}
% \end{table}

% \newpage

\section{Реализация на Python}

\newpage

\subsection{Graph}

\lstinputlisting[
    language=Python,
    name=graph.py
]
{../../src/graph.py}

\newpage

\subsection{Dijkstra's algorithm}

\lstinputlisting[
    language=Python,
    name=Dijkstra_s_algorithm.py
]
{../../src/Dijkstra_s_algorithm.py}

\begin{lstlisting}[
    name=Console out
]
Граф смежности
[[inf inf inf inf inf inf inf inf]
 [inf inf  3.  7.  4. inf inf inf]
 [inf  3. inf  5.  2. inf inf 10.]
 [inf  7.  5. inf inf inf  4. inf]
 [inf  4.  2. inf inf  4. inf inf]
 [inf inf inf inf  4. inf  5.  6.]
 [inf inf inf  4. inf  5. inf  7.]
 [inf inf 10. inf inf  6.  7. inf]]

Алгоритм Дейкстры
Стоимость пути из начальной вершины до остальных:
1  >  1  =  0.0
1  >  2  =  3.0
1  >  3  =  7.0
1  >  4  =  4.0
1  >  5  =  8.0
1  >  6  =  11.0
1  >  7  =  13.0

Граф смежности
[[inf inf inf inf inf inf inf inf]
 [inf inf  3.  7.  4. inf inf inf]
 [inf  3. inf  5.  2. inf inf 10.]
 [inf  7.  5. inf inf inf  4. inf]
 [inf  4.  2. inf inf  4. inf inf]
 [inf inf inf inf  4. inf  5.  6.]
 [inf inf inf  4. inf  5. inf  7.]
 [inf inf 10. inf inf  6.  7. inf]]
\end{lstlisting}

\newpage

\subsection{Floyd Warshell algorithm}

\lstinputlisting[
    language=Python,
    name=Floyd_Warshell_algorithm.py
]
{../../src/Floyd_Warshell_algorithm.py}

\begin{lstlisting}[
    name=Console out
]
Граф смежности
[[inf inf inf inf inf inf inf inf]
 [inf inf  3.  7.  4. inf inf inf]
 [inf  3. inf  5.  2. inf inf 10.]
 [inf  7.  5. inf inf inf  4. inf]
 [inf  4.  2. inf inf  4. inf inf]
 [inf inf inf inf  4. inf  5.  6.]
 [inf inf inf  4. inf  5. inf  7.]
 [inf inf 10. inf inf  6.  7. inf]]

Алгоритм Флоида Уоршала
1  >  1  =  inf
1  >  2  =  3.0
1  >  3  =  7.0
1  >  4  =  4.0
1  >  5  =  8.0
1  >  6  =  11.0
1  >  7  =  13.0

Граф смежности
[[inf inf inf inf inf inf inf inf]
 [inf inf  3.  7.  4.  8. 11. 13.]
 [inf  3. inf  5.  2.  6.  9. 10.]
 [inf  7.  5. inf  7.  9.  4. 11.]
 [inf  4.  2.  7. inf  4.  9. 10.]
 [inf  8.  6.  9.  4. inf  5.  6.]
 [inf 11.  9.  4.  9.  5. inf  7.]
 [inf 13. 10. 11. 10.  6.  7. inf]]
\end{lstlisting}

\newpage

\subsection{Kruskal's algorithm}

\lstinputlisting[
    language=Python,
    name=Kruskal_s_algorithm.py
]
{../../src/Kruskal_s_algorithm.py}

\begin{lstlisting}[
    name=Console out
]
Граф смежности
[[inf  5.  8. inf 11. inf inf]
 [ 5. inf  4.  8. inf inf inf]
 [ 8.  4. inf inf inf 10. inf]
 [inf  8. inf inf inf  2.  6.]
 [11. inf inf inf inf  7.  4.]
 [inf inf 10.  2.  7. inf  5.]
 [inf inf inf  6.  4.  5. inf]]

Алгоритм Краскаля
Ребро: Вес
3  -  5  :  2.0
1  -  2  :  4.0
4  -  6  :  4.0
0  -  1  :  5.0
5  -  6  :  5.0
1  -  3  :  8.0
Mинимальная сумма равна:  28.0

Граф смежности
[[inf inf inf inf 11. inf inf]
 [inf inf inf inf inf inf inf]
 [ 8. inf inf inf inf 10. inf]
 [inf  8. inf inf inf inf inf]
 [11. inf inf inf inf inf inf]
 [inf inf 10. inf inf inf inf]
 [inf inf inf inf inf inf inf]]
\end{lstlisting}

\newpage

\subsection{Prim's algorithm}

\lstinputlisting[
    language=Python,
    name=Prim_s_algorithm.py
]
{../../src/Prim_s_algorithm.py}

\begin{lstlisting}[
    name=Console out
]
Граф смежности
[[inf  5.  8. inf 11. inf inf]
 [ 5. inf  4.  8. inf inf inf]
 [ 8.  4. inf inf inf 10. inf]
 [inf  8. inf inf inf  2.  6.]
 [11. inf inf inf inf  7.  4.]
 [inf inf 10.  2.  7. inf  5.]
 [inf inf inf  6.  4.  5. inf]]

Остовные деревья
Алгоритм Прима
Ребро : Вес
0  -  0  :  inf
0  -  1  :  5.0
1  -  2  :  4.0
1  -  3  :  8.0
3  -  5  :  2.0
5  -  6  :  5.0
6  -  4  :  4.0

Граф смежности
[[inf  5.  8. inf 11. inf inf]
 [ 5. inf  4.  8. inf inf inf]
 [ 8.  4. inf inf inf 10. inf]
 [inf  8. inf inf inf  2.  6.]
 [11. inf inf inf inf  7.  4.]
 [inf inf 10.  2.  7. inf  5.]
 [inf inf inf  6.  4.  5. inf]]
\end{lstlisting}

\newpage

% = = = = =

\section{Реализация на C++}

\newpage

\subsection{DFS}

\lstinputlisting[
    language=C,
    name=dfs.hpp,
]
{../../src/CodeBlocks_3-semestr-DiscreteMath_lab-3/dfs.hpp}

\lstinputlisting[
    language=C,
    name=dfs.cpp,
]
{../../src/CodeBlocks_3-semestr-DiscreteMath_lab-3/dfs.cpp}

\begin{lstlisting}[
    name=Console out
]
Обход графа в ширину начиная с 0 вершины: 
1 2 3 4 
Число компонент связности: 3
\end{lstlisting}

\newpage

\subsection{BFS}

\lstinputlisting[
    language=C,
    name=bfs.hpp,
]
{../../src/CodeBlocks_3-semestr-DiscreteMath_lab-3/bfs.hpp}

\lstinputlisting[
    language=C,
    name=bfs.cpp,
]
{../../src/CodeBlocks_3-semestr-DiscreteMath_lab-3/bfs.cpp}

\begin{lstlisting}[
    name=Console out
]
Обход в глубину графа:
1 2 3 4 5 6 7 8 9 
Число компонент связности: 3
\end{lstlisting}

\newpage

\subsection{Floyd}

\lstinputlisting[
    language=C,
    name=floyd.hpp,
]
{../../src/CodeBlocks_3-semestr-DiscreteMath_lab-3/floyd.hpp}

\lstinputlisting[
    language=C,
    name=floyd.cpp,
]
{../../src/CodeBlocks_3-semestr-DiscreteMath_lab-3/floyd.cpp}

\begin{lstlisting}[
    name=Console out
]
0    5    4    8    9   11   13 
5    0    9    3    6    6    8 
4    9    0   11    5    7   10 
8    3   11    0    8    4    7 
9    6    5    8    0    4    7 
11    6    7    4    4    0    3 
13    8   10    7    7    3    0 
\end{lstlisting}

\newpage

\subsection{Kruskal}

\lstinputlisting[
    language=C,
    name=Kruskal.hpp,
]
{../../src/CodeBlocks_3-semestr-DiscreteMath_lab-3/Kruskal.hpp}

\lstinputlisting[
    language=C,
    name=Kruskal.cpp,
]
{../../src/CodeBlocks_3-semestr-DiscreteMath_lab-3/Kruskal.cpp}

\begin{lstlisting}[
    name=Console out
]
Минимальное остовное дереро (алгоритм Крускала):
Ребро : Вес
a - b : 2
b - e : 3
a - c : 4
d - f : 4
a - f : 5
d - g : 6
\end{lstlisting}

\newpage

\subsection{Prim}

\lstinputlisting[
    language=C,
    name=prim.hpp,
]
{../../src/CodeBlocks_3-semestr-DiscreteMath_lab-3/prim.hpp}

\lstinputlisting[
    language=C,
    name=prim.cpp,
]
{../../src/CodeBlocks_3-semestr-DiscreteMath_lab-3/prim.cpp}

\begin{lstlisting}[
    name=Console out
]
Минимальное остовное дереро (алгоритм Прима):
Ребро : Вес
a - b : 2
b - e : 3
a - c : 4
a - f : 5
f - d : 4
d - g : 6
\end{lstlisting}

\newpage

\subsection{Pruefer}

\lstinputlisting[
    language=C,
    name=pruefer.hpp,
]
{../../src/CodeBlocks_3-semestr-DiscreteMath_lab-3/pruefer.hpp}

\lstinputlisting[
    language=C,
    name=pruefer.cpp,
]
{../../src/CodeBlocks_3-semestr-DiscreteMath_lab-3/pruefer.cpp}

\begin{lstlisting}[
    name=Console out
]
Исходное дерево:
0 1
0 2
1 3
1 4
2 5
2 6
4 7
7 8
7 9
9 10
9 11
11 12
Код Прюфера: 1 2 2 0 1 4 7 7 9 9 11 
Декодированное дерево:
3 1
5 2
6 2
2 0
0 1
1 4
4 7
8 7
7 9
10 9
9 11
11 12
\end{lstlisting}

\newpage

\subsection{Main}

\lstinputlisting[
    language=C,
    name=main.hpp,
]
{../../src/CodeBlocks_3-semestr-DiscreteMath_lab-3/main.hpp}

\lstinputlisting[
    language=C,
    name=main.cpp,
]
{../../src/CodeBlocks_3-semestr-DiscreteMath_lab-3/main.cpp}

\newpage

% \newpage

% Список использованных источников
\begin{thebibliography}{9}
	\bibitem{prim} 
        Алгоритм Прима
        \\\url{https://gist.github.com/ernado/5909088}
        
        \bibitem{krusk} 
        Kruskal's Algorithm
        \\\url{https://gist.github.com/hayderimran7/09960ca438a65a9bd10d0254b792f48f}

        \bibitem{krusk2} 
        Алгоритм Крускала (простая реализация для матрицы смежности) | Портал информатики для гиков
        \\\url{http://espressocode.top/kruskals-algorithm-simple-implementation-for-adjacency-matrix/}

        \bibitem{deikstr} 
        Алгоритм Дейкстры нахождения кратчайшего пути
        \\\url{https://prog-cpp.ru/deikstra/}
\end{thebibliography}
